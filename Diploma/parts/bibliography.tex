\begin{thebibliography}{99}                    % Список литературы
\addcontentsline{toc}{section}{Список литературы}

\bibitem{sauter}
F. Sauter, Z. Phys. \textbf{69}, 742 (1931); \textbf{73}, 547
(1931).

\bibitem{zommerfeld}
А. Зоммерфельд, \emph{Строение атома и спектры}, т. II, ГИТТЛ,
Москва (1956).

\bibitem{heisenberg}
W. Heisenberg and H. Euler, Z.Phys. \textbf{98}, 714 (1936).

\bibitem{schwinger}
J. Schwinger, Phys. Rev. \textbf{82}, 664 (1951).

\bibitem{vanyashin}
В. С. Ваняшин, М. В. Терентьев, ЖЭТФ \textbf{48}, 565 (1965).

\bibitem{bunkin}
Ф. В. Бункин, И. И. Тугов, ДАН СССР \textbf{187}, 541 (1969).

\bibitem{brezin}
E. Brezin and C. Itzykson, Phys. Rev. D \textbf{2}, 1191 (1970).

\bibitem{popov1}
В. С. Попов, Письма ЖЭТФ \textbf{13}, 185 (1971); ЖЭТФ \textbf{34},
709 (1972).

\bibitem{popov2}
В.С. Попов, Письма ЖЭТФ \textbf{18}, 255 (1973) ЯФ \textbf{19}, 584
(1974).

\bibitem{narozhny1}
Н. Б. Нарожный, А. И. Никишов, ЖЭТФ \textbf{38}, 427 (1974).

\bibitem{mostepanenko}
В. М. Мостепаненко, В. М. Фролов, ЯФ \textbf{19}, 451 (1974).

\bibitem{marinov}
M. S. Marinov and V. S. Popov, Fortschr. Phys. \textbf{25}, 373
(1977).

\bibitem{grib}
А. А. Гриб, С. Г. Мамаев, В. М. Мостепаненко, \emph{Квантовые
эффекты в интенсивных внешних полях}, Энергоатомиздат, Москва
(1988).

\bibitem{ringwald}
A. Ringwald, Phys. Lett. B \textbf{510}, 107 (2001); E-print
archives hep-ph/01112254, hep-ph/0304139.

\bibitem{popov3}
В. С. Попов, Письма в ЖЭТФ \textbf{74}, 133 (2001); Phys. Lett. A
\textbf{298}, 83 (2002); ЖЭТФ \textbf{121}, 1235 (2002).

\bibitem{bula}
C. Bula, C. Bamber, D.L. Burke et al., Phys. Rev. Lett. \textbf{76},
3116 (1996).

\bibitem{burke}
D. L. Burke, S. C. Berridge, C. Bula et al., Phys. Rev Lett.
\textbf{79}, 1626 (1997).

\bibitem{tajima}
T. Tajima and G. Mourou, Phys. Rev. ST-AB \textbf{5}, 031301 (2002).

\bibitem{bulanov}
S. V. Bulanov, T. Zh. Esirkepov, and T. Tajima, Phys. Rev. Lett.
\textbf{91}, 085001 (2003).

\bibitem{volkov}
D. M. Volkov, Z. Phys. \textbf{94}, 250 (1935).

\bibitem{berestecky}
В. Б. Берестецкий, Е. М. Лифшиц, Л. П. Питаевский, \emph{Квантовая
электродинамика}, Физматлит, Москва (2001).

\bibitem{ritus}
В. И. Ритус, А. И. Никишов, \emph{Квантовая электродинамика явлений
в интенсивном поле}, Труды ФИАН \textbf{111}, (1979).

\bibitem{narozhny2}
Н. Б. Нарожный, М. С. Фофанов, ЖЭТФ \textbf{117}, 867 (2000).

\bibitem{narozhny3}
N. B. Narozhny and M. S. Fofanov, Phys. Lett. A \textbf{295}, 87
(2002).

\bibitem{malka}
G. Malka, E. Lefebvre, and J. L. Miquel, Phys. Rev. Lett.
\textbf{78}, 3314 (1997).

\bibitem{narozhny4}
N. B. Narozhny, S. S. Bulanov, V. D. Mur, and V. S. Popov, Phys.
Lett. A \textbf{330}, 1 (2004).

\bibitem{narozhny5}
Н. Б. Нарожный, С. С. Буланов, В. Д. Мур, В. С. Попов, Письма ЖЭТФ
\textbf{80}, 434 (2004).

\bibitem{narozhny6}
Н. Б. Нарожный, А. И. Никишов, ТМФ \textbf{26}, 16 (1976).

\bibitem{bulanov2}
С.С. Буланов, Н.Б. Нарожный, В. Д. Мур, В. С. Попов, ЖЭТФ
\textbf{129}, 14 (2006).

\end{thebibliography}
