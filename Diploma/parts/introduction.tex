\begin{center}
\LARGE\bf{Введение}
\end{center}

Текст введения. Например:

Эффект рождения электрон-позитронных пар из вакуума под действием
электрического поля впервые обсуждался, по-видимому, в работах
Заутера \cite{sauter} в связи с так называемым парадоксом Клейна
(см., например, \cite{zommerfeld}). Вероятность перехода
вакуум-вакуум, которая в присутствии постоянного однородного
электромагнитного поля отлична от единицы за счёт эффекта рождения
$e^+ e^-$-пар, в главном приближении была найдена Гейзенбергом и
Эйлером \cite{heisenberg}, точные формулы в случаях вакуума
заряженных частиц со спином 0 и 1/2 получены Швингером
\cite{schwinger}, а в случае векторных бозонов --- Ваняшиным и
Терентьевым \cite{vanyashin}. Вероятность рождения $e^+ e^-$-пар из
вакуума становится заметно отличной от нуля при напряжённости
постоянного электрического поля, близкой к характерному для
квантовой электродинамики (КЭД) значению
$$E_S=\frac{m^2c^3}{e\hbar}=1.32\cdot 10^{16}\,\,В/см $$                % Формула посередине без нумерации
(см. \cite{sauter,zommerfeld,heisenberg,schwinger}), при котором
электрическое поле на комптоновской длине
 $$l_C =\frac{\hbar}{mc}=3.86\cdot 10^{-11}\,\,см$$
совершает над электроном работу $mc^2$. Постоянное поле такой
напряжённости вряд ли может быть получено в лабораторных условиях.
Поэтому многие авторы сосредоточились на теоретическом исследовании
процесса рождения пар под действием переменных во времени
электрических полей
\cite{bunkin,brezin,popov1,popov2,narozhny1,mostepanenko,marinov,grib,ringwald,popov3},
хотя и в этом случае надежды на достижение напряжённостей порядка
$E_S$ до последнего времени казались весьма призрачными.

\begin{center}
$I \sim I_S=(c/4\pi)E_S ^2=4.65\cdot 10^{29}$ Вт/см$^2$.                % Другой способ записать формулу посередине без нумерации
\end{center}

............................................................

Однако существуют процессы, для описания которых модель плоской
волны использовать невозможно. В частности, плоская электромагнитная
волна произвольной интенсивности и спектрального состава не рождает
$e^+ e^-$-пар из вакуума \cite{schwinger}, поскольку оба инварианта
электромагнитного поля плоской волны
\[\mathcal F=(\mathbf E^2-\mathbf H^2)/2,\quad\mathcal G=(\mathbf E\cdot\mathbf H)\]    % Третий способ -"-"- и пример каллиграфических букв
равны нулю. Поэтому в настоящей работе для описания
электромагнитного поля фокусированной волны использована
реалистическая трёхмерная модель, предложенная в работе
\cite{narozhny2}. Эта модель основана на точном решении уравнений
Максвелла и была успешно применена в работе \cite{narozhny3} для
количественного объяснения эффекта анизотропии углового
распределения электронов, ускоренных интенсивным лазерным импульсом,
который наблюдался в эксперименте \cite{malka}. Сразу же отметим,
что использование суперпозиции двух фокусированных импульсов
позволяет обнаружить рождение $e^+ e^-$-пар при интенсивностях,
значительно меньших, чем в случае одиночного импульса
\cite{narozhny4,narozhny5}.

\begin{equation}                % Формула с нумерацией и лейблом "pairs" для ссылки на неё с помощью (\ref{pairs})
 \label{pairs}
 \begin{aligned}                % Нужно для набора многострочных формул; в отличие от array, сохраняет размер символов в дробях и кое-что ещё...
  N=\frac{e^2E_S^2}{4\pi^2\hbar^2c}\int\limits_{V}dV\int\limits_{0}^{\tau}dt\,
  \epsilon\eta\, \mathrm{cth}\,
  \frac{\pi\eta}{\epsilon}\exp\left(-\frac{\pi}{\epsilon}\right).
 \end{aligned}
\end{equation}
Здесь
$$\epsilon=\mathcal E/E_S,\quad\eta=\mathcal H/E_S$$
--- приведённые поля, а $\mathcal E$ и $\mathcal H$ --- инварианты,
имеющие смысл напряжённостей электрического и магнитного поля в той
системе отсчёта, где они параллельны:
