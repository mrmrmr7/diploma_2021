
\section{Раздел 2}

Текст раздела


\subsection{Подраздел 1}

        % Пример таблицы. \multirow{x}{Ycm} позволяет объединить x строк в столбце, длину которого задаём равной Y см

\begin{table}[t]
\caption{\label{table1}Среднее число пар, рождённых одиночным
(слева) и двумя сталкивающимися (справа) циркулярно-поляризованными
импульсами $e$-типа из вакуума, $\Delta=0.1$}
\begin{center}
\begin{tabular}{|c|c|c|c|c|c|}
\hline \multirow{2}{2cm}{$I\cdot10^{-28}$,
Вт/см$^2$}&\multirow{2}{2cm}{\quad$E_0/E_S$}&
\multirow{2}{2.5cm}{$\qquad\,
N$}&\multirow{2}{2.5cm}{$I\cdot10^{-26}$, Вт/см$^2$}&
\multirow{2}{2.5cm}{\quad$E_0/E_S$}&\multirow{2}{2.5cm}{$\qquad\, N$}\\
&&&&&\\
\hline   0.6   &  0.203 &    1.94(-5) &1.0     &0.0262   & 2.36(-8)\\
\hline   0.8   &  0.234 &    5.57(-2) &1.5     &0.0321   & 3.12(-3)\\
\hline   1.0   &  0.262 &       13.4  &2.0     &0.0371   &    3.85\\
\hline   1.5   &  0.321 &     7.57(4) &2.5     &0.0414   &  5.20(2)\\
\hline   2.0   &  0.371 &     1.42(7) &3.0     &0.0454   &  2.01(4)\\
\hline   2.5   &  0.414 &     5.29(8) &4.0     &0.0524   &  3.59(6)\\
\hline   3.0   &  0.454 &     7.89(9) &5.0     &0.0586   &  1.33(8)\\
\hline   4.0   &  0.524 &    3.70(11) &6.0     &0.0642   &  1.95(9)\\
\hline   5.0   &  0.586 &    5.35(12) &7.0     &0.0693   & 1.61(10)\\
\hline   6.0   &  0.642 &    4.05(13) &8.0     &0.0741   & 8.94(10)\\
\hline   8.0   &  0.741 &    7.17(14) &9.0     &0.0786   & 3.75(11)\\
\hline  10.0   &  0.829 &    5.33(15) &10.0    &0.0829   & 1.28(12)\\
\hline
\end{tabular}
\end{center}
\end{table}

\subsection{Подраздел 2}

Текст подраздела