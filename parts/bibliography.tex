\begin{thebibliography}{99}
\addcontentsline{toc}{section}{Список использованной литературы}

\bibitem{menshikov}
М. В. Меньшиков, С. А. Молчанов, А. Ф. Сидоренко. \emph{Теория перколяции и некоторые приложения, Итоги науки и техн}, т. XXIV, Сер. Теор. вероятн. Мат. стат. Теор. кибернет.,   \textbf{53-57}, 110 (1986).


\bibitem{buzmakova}
М. M. Бузмакова. \emph{Перколяция сфер в континууме}, т. XII. Сер. Математика. Механика. Информатика, \textbf{48-55}, 49 (2012).

\bibitem{tarasevich}
Ю. Ю. Тарасевич. \emph{Перколяция: теория, приложения, алгоритмы}. Едиториал УРСС, \textbf{112} (2002). 

\bibitem{schlovsky}
Б. И. Шкловский, А. Л. Эфрос \emph{Электронные свойства легированных полупроводников}. Наука, \textbf{416} (1979). 

\bibitem{abrikosov}
A. A. Abrikosov, \emph{Spin glasses with short range interaction}.  т. 29. Adv. Phys. \textbf{869-946} (1980). 

\bibitem{mersen}
M. Malsumoto, \emph{Mersenne twister: A 623-dimensionally equidistributed uniform pseudorandom number generator}. ACM Trans, on Modeling and Computer Simulations.т. 8. \textbf{3-30} (1998). 

\bibitem{hoshen}
J. Hoshen, R. Kopelman. \emph{Percolation and cluster distribution. I. Cluster multiple labeling technique and critical concentration algorithm}. 11 Physical Review. т. 14,
№ 8. \textbf{3438-3445} (1976). 



\end{thebibliography}
