
\section{Заключение}

Текст заключения

Рассмотренные случаи циркулярно-поляризованной и
линейно-поляризованной волн позволяют сделать утверждение, что для
процесса рождения $e^+e^-$-пар эффективнее сталкивать лазерные
импульсы, а не использовать одиночные, поскольку в этом случае порог
рождения пар составляет величину $I \sim 10^{26}$ Вт/см$^2$, что на
один-два порядка меньше, чем в случае одиночного импульса. Более
того, в данной работе впервые показано, что эффективнее использовать
лазерные импульсы с линейной поляризацией. Это вызвано в первую
очередь тем, что при одной и той же интенсивности лазерного импульса
напряжённость электрического поля линейно-поляризованной волны в
$\sqrt{2}$ больше, чем у волны с циркулярной поляризацией. Как
показывают расчёты, логарифм числа частиц как функция $E_0$ -
напряжённости каждого поля по отдельности, - практически одинаков
для обеих поляризаций. Но если рассматривать $\lg N$ как функцию
интенсивности лазерного импульса, то получится, что поле с линейной
поляризацией рождает на несколько порядков пар больше, что довольно
существенно. Это означает, что на эксперименте эффективнее будет
использовать именно линейно-поляризованное электромагнитное поле.
