\begin{center}
\LARGE\bf{Введение}
\end{center}

Теория протекания или теория перколяции – раздел науки, изучающий образование связанных структурированных объектов в неупорядоченных неструктурированных средах. На основании теории перколяции возможно построить модели применимые как в математике, химии, биологии так и в физике. В математике теория перколяции описывается, опираясь на теорию графов, а в физических задачах данная теория применяется для изучения процессов, происходящих. в неоднородных средах со случайными, но фиксированными в пространстве и времени свойствами. К таким задачам относится задача эффективной электропроводимости вещества, состоящего из нескольких материалов. Также на данный момент ведутся исследования применения теории перколяции для управления потоками данных в информационных сетях на транспорте, в механике жидкостей и плазмы и так далее.

В данной работе продемонстрирована реализация алгоритмов, позволяющих генерировать области с континуальным расположением элементов различной формы в двухмерной и трёхмерной области, а также задачи анализа данной области. Цель данной работы – создание библиотеки для современных языка программирования Python с большим спектром возможностей от задания количества элементов до процента заполненности этими элементами области.