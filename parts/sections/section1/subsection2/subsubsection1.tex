\subsubsection{Случайно распределенные ансамбли}
Для расположения элементов случайным образом используются несколько различнх подходов. В данной работе рассмотрены метод Монте-Карло а также метод случайных блужданий.

Рассмотрим сначала метод Монте-Карло и его адаптацию к задаче генерации поля с ансамблем случайно-распределенных элементов.
Обобщённый алгоритм упаковки элементов выглядит следующим образом:
\begin{enumerate}
    \item Координаты центра первого элемента генерируются и записываются в массивы координат, а элементу присваивается номер $i=1$.
    \item Для каждого следующего элемента  генерируются координаты центра.
    \item Проверяется, есть ли в области, равной удвоенному характерному размеру генерируемых элементов, какие-либо генерируемые ранее элементы.
    \item Если такие находятся, то идет проверка на пересекаемость нового элемента с каждым элементом находящимся в этой области.
    \item Если сфера пересекается с какой-либо из таких сфер, она отвергается: $i$ остается прежним и переходим ко второму пункту.
    \item Если сфера не пересекается с какой-либо из таких элементов, она принимается: ее координаты записываются в массивы, элементу присваивается номер $i$.
    \item И так далее до тех пор, пока $i$ не станет равным $n$. 
\end{enumerate}

Однако данный алгоритм обладает значительным ограничением при генерации области на текущих вычислительных мощностях. Для генерации области с заданной плотностью заполненности элементами больше 50-ти процентов, время такой генерации значительно превышает время, которое можно получить альтернативными методами. Лучшим альтернативным методом является разработанный метод случайных блужданий:
\begin{enumerate}
    \item Выбирается некоторая сетка, на которой генерируются элементы в пространстве.
    \item Для каждого элемента генерируются объект с заранее заданными параметрами.
    \item Для каждого элемента производится попытка сместить элемент относительно предыдущего положения (для первого шага предыдущее положение - это положение на узлах сетки).
    \item Проверяется, есть ли в области, равной удвоенному характерному размеру генерируемых элементов, какие-либо генерируемые ранее элементы.
    \item Если такие находятся, то идет проверка на пересекаемость нового элемента с каждым элементом находящимся в этой области.
    \item Если элемент пересекается с каким-либо из таких элементов, он отвергается: $i$ остается прежним и переходим ко второму пункту.
    \item Если элемент не пересекается с каким-либо из таких элементов, он принимается: его координаты записываются в массивы, элементу присваивается номер $i$.
    \item И так далее до тех пор, пока $current_shuffle_count$ не станет равным $shuffle_count$. 
\end{enumerate}
Данный метод позволяет добиться большей плотности элементов. Максимальная плотно