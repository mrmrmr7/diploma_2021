\subsection{Современные направления исследований}

Чем больше перколяция развивается как раздел физики, тем больше обнаруживается областей в которых можно применить перколяцию.
Одним из направлений исследования на данный момент является электропроводящие композиционные материалы. 
Эффективная электропроводность гетерогенной системы не совпадает со средней по объему электропроводностью. 
Установление ее связи со значениями удельных электропроводностей и объемными концентрациями фаз является одной из важных задач физики гетерогенных систем. 
Аналитическое решение задачи проводят в рамках теорий эффективной среды и протекания (перколяции). 
При изучении композиционных материалов в первом приближении полагают, что бесконечный токопроводящий кластер возникает только при концентрациях наполнителя, больших порога перколяции, и состоит из однотипных объектов — частиц электропроводящей фазы. 
Такому строению композитных материаллов отвечают простейшие модельные объекты теории протекания, например, решетки из резисторов с определенным и постоянным сопротивлением или же связанные области пространства с постоянным значением удельной электропроводности. 
Однако для получения более точных результатов следует перейти от данного приближения к постановке континуальной задаче, при которой отсутствует строго заданная сетка.

Одним из широко распространенных подходов для описания процесса гелеобразования является теория перколяции. 
Она изучает образование связанных объектов в неупорядоченной среде. \cite{buzmakova} 
С помощью теории перколяции описаны многие физические, химические и другие процессы. 
Известны математические модели процесса гелеобразования, основанные на методах теории перколяции [11–15]. 
Используются решеточные и континуальные модели. 
В решеточных моделях занятые узлы рассматривают в качестве молекул растворенного вещества, а пустые — в качестве молекул растворителя.
Однако в большинстве случаев решеточные модели оказываются слишком упрощенными для описания реальных систем, так как, во-первых, молекулы растворенного вещества, как правило, не являются точечными объектами, во-вторых, координаты молекул в реальных системах являются непрерывными, а не дискретными.
Кроме того, использование решеточных моделей предполагает наличие у моделируемой системы дальнего порядка — трансляционной симметрии. 
В жидкостях имеется только ближний порядок, поэтому искусственная дискретизация области привносит в модель неприсущие ей свойства. 
Если говорить конкретно о процессе гелеобразования белковых молекул, то характерный размер молекулы человеческого сывороточного альбумина составляет величину порядка $10^{-8}$ м (см., например, [16]), что на два порядка превышает характерные расстояния между молекулами воды. 
Поэтому молекулы альбумина в водном растворе более естественно рассматривать как частицы, находящиеся в непрерывной среде. 
Таким образом, описание процесса гелеобразования с помощью моделирования континуальной перколяционной системы представляется более адекватным по сравнению с решеточными моделями. 
Под континуальной перколяционной системой понимается пространственная трехмерная система с вещественными координатами.