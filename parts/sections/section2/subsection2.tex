\subsection{Критерии случайности ансамбля}

Для того, чтобы определить, является-ли расположение элементов ансамбля случайным, следует произвести генерацию элементов кластера $N_{generations}$ раз с одинаковым количеством элементов $N_{elements}$, присвоив каждому элементу кластера собственный индекс $e_{i}$, а затем исследовать закономерности расположения элементов в рамках каждого $i \in [0, N_{elements}]$, при фиксированном $j \in [0, N_{generations}]$ 

Если выборка значений случайна, то значение каждого ее элемента не должно зависеть от величины предшествующего и последующего членов. Для проверки этой независимости используется статистика являющаяся коэффициентом корреляции первого порядка
между элементами первичной выборки 
$$
r_{1, n}=\frac{n \sum_{i=1}^{n-1} x_{i} x_{i+1}-\left(\sum_{i=1}^{n} x_{i}\right)^{2}+n x_{1} x_{n}}{n \sum_{i=1}^{n} x_{i}^{2}-\left(\sum_{i=1}^{n} x_{i}\right)^{2}}
$$
являющаяся коэффициентом корреляции первого порядка
между элементами первичной выборки $ \left(x_{1}, \ldots, x_{n}\right) $ и элементами выборки, полученной из нее сдвигом на одну единицу $\left(x_{2}, x_{3}, \ldots, x_{n}, x_{1}\right)$. Величину $r_{1, n}$,
можно считать распределенной асимптотически нормально со средним $M\left(r_{1, n}\right)$ и дисперсией $D\left(r_{1, n}\right)$, где $M\left(r_{1, n}\right)=-\frac{1}{n-1}$ \newline
$D\left(r_{1, n}\right)=\frac{n(n-3)}{(n+1)(n-1)^{2}}$ Поэтому в качестве критерия случайности может рассматриваться нормализованная статистика 
$$
r_{1, n}^{*}=\frac{\left|r_{1, n}-M\left(r_{1, n}\right)\right|}{\sqrt{D\left(r_{1, n}\right)}}
$$ 
Гипотеза о случайности отклоняется при $\left|r_{1, n}^{*}\right|>\boldsymbol{u}_{1-\frac{\alpha}{2}}$.

В нашем случае можно проводить такую проверку между координатами расположения элементов. Для этого рассмотрим критерий автокорреляции применимо к \textit{x} координате, а также к \textit{y} координате каждого из элементов. В результате если корреляции между конечными значениями координат \textit{x} и \textit{y} для каждой из частиц нет, то можно утверждать, что и корреляции между положением окружностей также нет.

Результаты проверки критерием автокорреляции показали, что при одинаковом начальном расположении элементов в результате перемешивания, окружности оказываются в различных местах.

Проводя эксперимент с определенным количеством окружностей и сохранив конечные положения каждой окружности в каждой генерации, можно провести тест автокорреляции. При значении квантиля стандартного нормального распределения $u_{1-\alpha/2}$, генерации случайных ансамблей с различным количеством окружностей (от $10$ до $70$) показало, что гипотеза о случайности не отклоняется c вероятностью в $0.95$. 

Данный метод проверки позволяет сделать заключение о том, что при генерации случайных ансамблей частиц методом плотнейших упаковок с последующим перемешиванием элементов из одинаковой начальной конфигурации получать различные конечные формации ансамблей. 

"Случайность" ансамбля также может повысить тот факт, что при проведении теста элементы специально индексировались в фиксированной последовательности. При настоящем исследовании генерирующихся кластеров такого можно не делать (то есть рассматривать все элементы как равные). 

\newcommand{\imgsize}{8cm}
\renewcommand{\imgsubdir}{autocorrelation}
\begin{figure}[h!]
    \begin{subfigure}{0.49\textwidth}
        \centering
        \includegraphics [width=8cm,height=\imgsize]
        {\imgdir/\imgsubdir/10c_10g_10ok.png}
        \caption{$10$ шаров, тест автокорреляции}
    \end{subfigure}
    \begin{subfigure}{0.49\textwidth}
        \centering
        \includegraphics [width=8cm,height=\imgsize]
        {\imgdir/\imgsubdir/30c_10g_29ok.png}
        \caption{$30$ шаров, тест автокорреляции}
    \end{subfigure}
    \begin{subfigure}{0.49\textwidth}
        \centering
        \includegraphics [width=8cm,height=\imgsize]
        {\imgdir/\imgsubdir/50c_10g_50ok.png}
        \caption{$50$ шаров, тест автокорреляции}
    \end{subfigure}
    \begin{subfigure}{0.49\textwidth}
        \centering
        \includegraphics [width=8cm,height=\imgsize]
        {\imgdir/\imgsubdir/70c_10g_65ok.png}
        \caption{$70$ шаров, тест автокорреляции}
    \end{subfigure}
    \caption{Расположение шаров}
    \label{fig:tight_packing}
\end{figure}

Для проверки как расположены элементы ансамбля проводится проверка на подчинение гипотезе о равномерном распределении элементов по каждой из осей. 

Для того чтобы проверить гипотезу о равномерном распределении $X$,т.е. по закону: $$f(x) = \frac{1}{b-a}$$ в интервале $(a,b)$
надо:
\begin{enumerate}
    \item 
        Найдём дисперсию и выборочное среднее:
        $$
        \bar{x}=\frac{1}{n} \sum x_{i} n_{i}
        $$
        $$
        D_{B}=\frac{1}{n} \sum\left(x_{i}-\bar{x}\right)^{2} n_{i}
        $$
    \item
        Оценить параметры $a$ и $b$ - концы интервала, в котором наблюдались возможные значения $X$, по формулам (через знак $*$ обозначены оценки параметров) $$a^{*}=\bar{x}-\sqrt{3} \sigma$$
        $$b^{*}=\bar{x}+\sqrt{3} \sigma$$
    \item 
        Найти плотность вероятности предполагаемого распределения 
        $$f(x) = \frac{1}{b* - a*}$$
    \item
        Найти теоретические частоты:
        $$
        n_{1}^{\prime}=n \frac{1}{73,763}\left(x_{1}-a\right)
        $$ 
        $$
        n_{i}^{\prime}=n \frac{1}{73,763}\left(x_{i+1}-x_{i}\right)
        $$
        $$
        n_{s}^{\prime}=n \frac{1}{73,763}\left(b-x_{s}\right)
        $$
    \item
        Сравнить эмпирические и теоретические частоты с помощью критерия Пирсона: 
        $$
        \chi^{2}=\sum \frac{\left(n_{i}-n_{i}^{\prime}\right)^{2}}{n_{i}^{\prime}}
        $$
        приняв число степеней свободы $k = s - 3$, где $s$ - число первоначальных интервалов выборки; если же было произведено объединение малочисленных частот, следовательно, и самих интервалов, то $s$ - число интервалов, оставшихся после объединения.
\end{enumerate}

В результате генерации и исследования ансамбля на равномерность распределения по пространству выявлено, что большинство генераций после перемешивания подходят под гипотезу равномерного распределения.

\begin{figure}[h]
    \centering
    \includegraphics[width=\textwidth]{\imgdir/uniform.png}
    \caption{Процент равномерных распределений среди генерации областей}
    \label{fig:uniform}
\end{figure}