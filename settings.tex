\documentclass[a4paper,12pt]{article} 
\usepackage{geometry}           % пакет для задания полей страницы командой \geometry
\geometry{left=3cm,right=1.5cm,top=2cm,bottom=2cm}
% \usepackage[cp1251]{inputenc}   % кодировка текста
\usepackage{mathtext}           % позволяет использовать русские буквы в формулах
\usepackage[T2A]{fontenc}       %пакет Т2А необходим для правильного отображения кириллицы и переноса слов
%\inputencoding{cp1251}          % тоже кодировка...
\usepackage[russian]{babel}     % языковой пакет - последний язык главный
%\usepackage[unicode]{hyperref}  %создаёт гиперссылки на список литературы в pdf-файле
\usepackage{amstext,amsmath,amssymb}            % пакеты для формул
\usepackage{bm}                 % boldmath - пакет для жирного шрифта
\usepackage[pdftex]{graphicx}   % пакет для включения рисунков в форматах png,pdf,jpg,mps,tif
                                % надо компилировать сразу в pdf
\usepackage{amsfonts}           % греческие символы и, возможно, что-то ещё
\usepackage{indentfirst}        % одинаковый отступ для первого параграфа и всего остального
\usepackage{cite}               % команда /cite{1,2,7,9} даёт ссылки
\usepackage{multirow}           % пакет для объединения строк в таблице: надо указать число строк и ширину столбца
\usepackage{array}              % нужен для создания таблиц

\usepackage{titlesec}
\usepackage{enumitem}
\usepackage{tabularx}

\usepackage{graphicx}
   
\usepackage{subcaption}   

\titleformat{\section}[block]{\centering\bfseries\huge}{Глава \thesection.}{1em}{}
\usepackage{caption}
\renewcommand{\thefigure}{\thesubsection.\arabic{figure}}
\DeclareCaptionLabelSeparator{bar}{\space--\space}
\captionsetup{labelsep=bar,justification=centering}

\linespread{1.3}                % полтора интервала. Если 1.6, то два интервала
\pagestyle{plain}               % номерует страницы
